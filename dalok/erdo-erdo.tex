\begin{dal}{Erdő, erdő\ldots}
\vsz{
  Erdő, erdő kerek erdő,

  De szép madár lakja kettő.

  Kék a lába, zöld a szárnya,

  Piros a rózsám orcája.}
  
\vsz{
  Olyan piros, mint a vér,

  Tőlem gyakran csókot kér.

  De én bizony nem adok,

  Inkább jól megátkozom.}

\vsz{
  Kilenc fia néma legyen,

  A tizedik leány legyen,

  Az is olyan csalfa legyen,

  Ország-világ híre legyen.}

\vsz{
  Kinek nincsen szeretője,

  Menjen ki a zöld erdőbe,

  Írja fel egy falevélre,

  Néki nincsen szeretője.}

  \vsz{
    Kinek nem jó itt lenn lakni,

    Menjen Mennyországba lakni,

    Építsen az égre házat,

    Ott nem éri semmi bánat
  }

  \vsz{
    Építsen az ég szélére,

    Ott nem éri semmiféle,

    Építsen az égre házat,

    Ott nem éri semmi bánat.
  }

  \vsz{
    Szeress, babám, csak nézd meg kit,

    Mert a szerelem az elvakít,

    Engemet is elvakított,

    Örökre megszomorított
  }

  \vsz{
    Olyan furcsa kedvem vagyon,

    A szőkét szeretem nagyon,

    A barna sem ellenségem,

    Most is ő a feleségem.
  }

  \vsz{
    Bonchidai sűrű berek,

    Ott terem az igaz gyerek.

    Ott termettem én magam is,

    Az én kedves galambom is.
  }

  \vsz{
    Bonchidai sűrű cserfa,

    Nem láttam a rózsám még ma.

    Nem is látom vasárnapig,

    Meghasad a szívem addig.
  }

  \vsz{
    Egek, egek, bús fellegek,

    Mondjátok meg, hogy haljak meg.

    Mondjátok meg, hogy haljak meg,

    Hogy a szívem szakadjon meg.
  }

  \vsz{
    Nyíljál meg föld, rejts el engem,

    Ne engedd, hogy reggelt érjek,

    Mert engement minden reggel

    Csak a búbánat serkent fel.
  }
\end{dal}

